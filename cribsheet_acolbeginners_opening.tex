\documentclass[14pt]{extarticle}
\usepackage[usenames,dvipsnames,svgnames,table]{xcolor}
\renewcommand{\familydefault}{\sfdefault}
% \DeclareSymbolFont{extraup}{U}{zavm}{m}{n}
% \DeclareMathSymbol{\varheart}{\mathalpha}{extraup}{86}
% \DeclareMathSymbol{\vardiamond}{\mathalpha}{extraup}{87}
\definecolor{lightgray}{gray}{0.95}
\usepackage{pifont}
\usepackage{dingbat}
\usepackage{graphicx}
\usepackage[letterspace=-50]{microtype}
\usepackage[a5paper,landscape]{geometry}



\usepackage{background}


\SetBgScale{1}
\SetBgAngle{0}
\SetBgColor{black}
\SetBgPosition{current page.south}
\SetBgVshift{10pt}
\SetBgContents{\tikz[remember picture,overlay]
        \node at ([yshift=0pt,xshift=0pt]current page.center) {\includegraphics[width=22cm]{blue.png}};    %% yshift and xshift for example only
    }

\usepackage{mathabx}

\begin{document}
\addtocounter{page}{5}
\newpage
\vspace*{\fill}
\begingroup
\centering
\section*{Opening 1NT}
\endgroup
Open 1NT if you have:\\
\begin{itemize}
\item a balanced hand, i.e. 4-3-3-3, 4-4-3-2 or 5-3-3-2 shape
\item 12, 13 or 14 HCP
\end{itemize}
\noindent \smallpencil \underline{Remember}: a balanced hand contains no singletons, no voids, no more than one doubleton.


\vspace*{\fill}
\begin{flushbottom}
 \begin{center}
 \tiny
 \textcopyright Andrew McIntosh
\end{center}
\end{flushbottom}


\newpage
\addtocounter{page}{1}

\vspace*{\fill}
\begingroup
\centering
\section*{Opening 1\textcolor{ForestGreen}{\ding{168}}/\textcolor{orange}{\ding{169}}/\textcolor{red}{\ding{170}}/\textcolor{blue}{\ding{171}}}
\endgroup
\noindent Open 1 of a suit when you cannot open 1NT, if you have:\\
\begin{itemize}
\item an unbalanced hand
\item a balanced hand with 15-19 HCP
\end{itemize}
\vspace{0.15in}
\noindent To open 1 of a suit you need:
\begin{itemize}
\item at least 12 HCP
\item 11HCP 11 HCP and a 6-card suit
\end{itemize}

\vspace*{\fill}
\begin{flushbottom}
 \begin{center}
 \tiny
 \textcopyright Andrew McIntosh
\end{center}
\end{flushbottom}


\newpage
\vspace*{\fill}
\begingroup
\centering
\section*{Opening 1\textcolor{ForestGreen}{\ding{168}}/\textcolor{orange}{\ding{169}}/\textcolor{red}{\ding{170}}/\textcolor{blue}{\ding{171}} - Which suit?}
\endgroup

\noindent How to choose the opening suit:\\
\begin{itemize}
\item the longest suit takes precedence
\item with two suits of the same length open the higher ranking \item with three suits of the same length, i.e. 4-1-4-4, open the suit below the singleton
\end{itemize}

\noindent \smallpencil \underline{Remember} the one exception: with exactly 4\textcolor{red}{\ding{170}} and 4\textcolor{blue}{\ding{171}}, open 1\textcolor{red}{\ding{170}}.

\vspace*{\fill}
\begin{flushbottom}
 \begin{center}
 \tiny
 \textcopyright Andrew McIntosh
\end{center}
\end{flushbottom}


\end{document}
