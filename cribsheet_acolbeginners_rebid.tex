\documentclass[14pt]{extarticle}
\usepackage[usenames,dvipsnames,svgnames,table]{xcolor}
\renewcommand{\familydefault}{\sfdefault}
% \DeclareSymbolFont{extraup}{U}{zavm}{m}{n}
% \DeclareMathSymbol{\varheart}{\mathalpha}{extraup}{86}
% \DeclareMathSymbol{\vardiamond}{\mathalpha}{extraup}{87}
\definecolor{lightgray}{gray}{0.95}
\usepackage{pifont}
\usepackage{dingbat}
\usepackage{graphicx}
\usepackage[letterspace=-50]{microtype}
\usepackage[a5paper,landscape]{geometry}



\usepackage{background}


\SetBgScale{1}
\SetBgAngle{0}
\SetBgColor{black}
\SetBgPosition{current page.south}
\SetBgVshift{10pt}
\SetBgContents{\tikz[remember picture,overlay]
        \node at ([yshift=0pt,xshift=0pt]current page.center) {\includegraphics[width=22cm]{orange.png}};    %% yshift and xshift for example only
    }

\usepackage{mathabx}


\begin{document}
\addtocounter{page}{16} 
\newpage

\vspace*{\fill}
\begingroup
\centering
\section*{Opener's rebid (A) - with a fit}

\endgroup
\noindent Opener's second call is known as their rebid.\\
\noindent When opener has a fit with responder he should support/raise.\\

\noindent For instance, the auction has started 1\textcolor{red}{\ding{170}} - 1\textcolor{blue}{\ding{171}}:\\
\begin{center}
$\begin{tabular}{|c|l|}
\hline
Rebid & Meaning\\
\hline
\rowcolor{lightgray}
2\textcolor{blue}{\ding{171}} & 13-15 points, 4\textcolor{blue}{\ding{171}}\\
\hline
3\textcolor{blue}{\ding{171}} &  16-18 points, 4\textcolor{blue}{\ding{171}}\\
\hline
\rowcolor{lightgray}
4\textcolor{blue}{\ding{171}} &  19 points, 4\textcolor{blue}{\ding{171}}\\
\hline\end{tabular}$\\\end{center}
\vspace{0.15in}
\noindent Responder can then decide the correct zone.

\vspace*{\fill}
\begin{flushbottom}
 \begin{center}
 \tiny
 \textcopyright Andrew McIntosh
\end{center}
\end{flushbottom}


\newpage

\vspace*{\fill}
\begingroup
\centering

\section*{Opener's rebid (B) - balanced}
\endgroup
\noindent With a balanced hand but no fit with responder, opener rebids NTs.\\
% 
% \begin{center}
% 
% $\begin{tabular}{|c|l|}
% \hline
% Rebid & Meaning\\
% \hline
% \rowcolor{lightgray}
% 1NT & 15-16 HCP, balanced\\
% 2NT & 17-18 HCP, balanced\\
% \rowcolor{lightgray}
% 3NT & 19 HPC, balanced\\
% \hline
% \end{tabular}
% $\\
% \end{center}

\noindent For instance, the auction has started 1\textcolor{ForestGreen}{\ding{168}} - 1\textcolor{red}{\ding{170}}:
\\
\begin{center}
$\begin{tabular}{|c|l|}
\hline
Rebid & Meaning\\
\hline
\rowcolor{lightgray}
1NT & 15-16 HCP, balanced\\
\hline
2NT &17-18 HCP, balanced\\
\hline
\rowcolor{lightgray}
3NT & 19 HCP, balanced\\
\hline
\end{tabular}$\\\end{center}
\vspace{0.15in}
\noindent Responder can then decide the correct zone.
\vspace*{\fill}
\begin{flushbottom}
 \begin{center}
 \tiny
 \textcopyright Andrew McIntosh
\end{center}
\end{flushbottom}




\vspace*{\fill}
\begingroup
\centering

\section*{Opener's rebid (C) - second suit}
\endgroup
\noindent If opener does not have a fit in responder's suit and has an unbalanced hand, they should show a second suit. This promises five or more cards in the first suit bid and four or more cards in the second. \\
% 
% \begin{center}
% $\begin{tabular}{|l|l|}
% \hline
% Rebid & Meaning\\
% \hline
% \rowcolor{lightgray}
% 2 of a new suit & 13-18 HCP, 5$^{+}$ cards in first suit\\
% jump in a new suit & 19 HCP, 5$^{+}$ cards in first suit\\
% \hline
% \end{tabular}
% $\\
% \end{center}

\noindent For instance, the auction has started 1\textcolor{orange}{\ding{169}} -  1\textcolor{red}{\ding{170}}:\\
\begin{center}
$\begin{tabular}{|c|l|}
\hline
Rebid & Meaning\\
\hline
\rowcolor{lightgray}
1\textcolor{blue}{\ding{171}} & 13-18 HCP, 5$^{+}$\textcolor{orange}{\ding{169}}, 4$^{+}$\textcolor{blue}{\ding{171}}\\
\hline
2\textcolor{ForestGreen}{\ding{168}} & 13-18 HCP, 5$^{+}$\textcolor{orange}{\ding{169}}, 4$^{+}$\textcolor{ForestGreen}{\ding{168}}\\
\hline
\rowcolor{lightgray}
2\textcolor{blue}{\ding{171}} & 19 HCP, 5$^{+}$\textcolor{orange}{\ding{169}}, 4$^{+}$\textcolor{blue}{\ding{171}}\\
\hline
\end{tabular}
$
\end{center}
\noindent \smallpencil\underline{Remember}: Jump if game zone values are present.
\vspace*{\fill}
\begin{flushbottom}
 \begin{center}
 \tiny
 \textcopyright Andrew McIntosh
\end{center}
\end{flushbottom}




\vspace*{\fill}
\begingroup
\centering

\section*{Opener's rebid (D) - without a fit}
\endgroup
\noindent If opener does not have a fit in responder's suit, or a balanced hand, or a second suit, they must repeat their own suit.\\
\begin{center}

$\begin{tabular}{|l|l|}
\hline
Rebid & Meaning\\
\hline
\rowcolor{lightgray}
2 of own suit & 13-15 points, 6$^{+}$ cards in suit\\
3 of own suit & 16-18 points, 6$^{+}$ cards in suit\\
\rowcolor{lightgray}
4 of own suit & 19 points, 6$^{+}$ cards in suit\\
\hline
\end{tabular}
$\\
\end{center}
\vspace{0.15in}
\noindent For instance, the auction has started 1\textcolor{red}{\ding{170}} - 1\textcolor{blue}{\ding{171}}:
\vspace{0.15in}
\begin{center}
$\begin{tabular}{|c|l|}
\hline
Rebid & Meaning\\
\hline
\rowcolor{lightgray}
2\textcolor{red}{\ding{170}} & 13-15 points, 6$^{+}$\textcolor{red}{\ding{170}}\\
\hline
3\textcolor{red}{\ding{170}} & 16-18 points, 6$^{+}$\textcolor{red}{\ding{170}}\\
\hline
\rowcolor{lightgray}
4\textcolor{red}{\ding{170}} & 19 points, 6$^{+}$\textcolor{red}{\ding{170}}\\
\hline
\end{tabular}
$
\end{center}

\vspace*{\fill}
\begin{flushbottom}
 \begin{center}
 \tiny
 \textcopyright Andrew McIntosh
\end{center}
\end{flushbottom}



\end{document}
