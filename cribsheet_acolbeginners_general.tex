\documentclass[14pt]{extarticle}
\usepackage[usenames,dvipsnames,svgnames,table]{xcolor}
\renewcommand{\familydefault}{\sfdefault}
% \DeclareSymbolFont{extraup}{U}{zavm}{m}{n}
% \DeclareMathSymbol{\varheart}{\mathalpha}{extraup}{86}
% \DeclareMathSymbol{\vardiamond}{\mathalpha}{extraup}{87}
\definecolor{lightgray}{gray}{0.95}
\usepackage{pifont}
\usepackage{dingbat}
\usepackage{graphicx}
\usepackage[letterspace=-50]{microtype}
\usepackage[a5paper,landscape]{geometry}



\usepackage{background}


\SetBgScale{1}
\SetBgAngle{0}
\SetBgColor{black}
\SetBgPosition{current page.south}
\SetBgVshift{10pt}
\SetBgContents{\tikz[remember picture,overlay]
        \node at ([yshift=0pt,xshift=0pt]current page.center) {\includegraphics[width=22cm]{yellow.png}};    %% yshift and xshift for example only
    }
\usepackage{mathabx}

\begin{document}
\newpage
\addtocounter{page}{3}
\vspace*{\fill}
\begingroup
\centering
\section*{Zones}
\endgroup

\noindent If a partnership has 33 points or more, it should play in the \textbf{slam} or grand slam zone (6\textcolor{ForestGreen}{\ding{168}} or higher).\\

\noindent If a partnership has between 25 and 32 points, it should play in the \textbf{game} zone (3NT, 4\textcolor{red}{\ding{170}}, 4\textcolor{blue}{\ding{171}}, 5\textcolor{ForestGreen}{\ding{168}}, 5\textcolor{orange}{\ding{169}}).\\

\noindent If a partnership has fewer than 25 points, it should play in as low a contract as possible (the \textbf{part score} zone).\\

\vspace*{\fill}
\begin{flushbottom}
 \begin{center}
 \tiny
 \textcopyright Andrew McIntosh
\end{center}
\end{flushbottom}







\newpage
\vspace*{\fill}
\begingroup
\centering

\section*{The game zone priority list}
\endgroup
\noindent When a partnership has between 25 and 32 points, it should play in the game zone, but which one is best? Follow this priority list:\\

\begin{itemize}
 \item Top priority: if you have a fit in a Major, play in 4\textcolor{red}{\ding{170}} or 4\textcolor{blue}{\ding{171}}
\item Second priority: if you do not have a fit in a Major, play in 3NT
\item The last resort: if you have two very unbalanced hands, with a fit in a minor, play in 5\textcolor{ForestGreen}{\ding{168}} or 5\textcolor{orange}{\ding{169}}, but that is generally too much like hard work.
\end{itemize}

\vspace*{\fill}
\begin{flushbottom}
 \begin{center}
 \tiny
 \textcopyright Andrew McIntosh
\end{center}
\end{flushbottom}


\newpage
\addtocounter{page}{2}
\vspace*{\fill}
\begingroup
\centering

\section*{Shortage points}
\endgroup
\noindent When you have found a \textbf{fit} (4$^{+}$ cards in partner's suit) responder may add shortage points:\\
\begin{itemize}
\item +1 point for each doubleton
\item +2 points for each singleton
\item +3 points for each void
\end{itemize}

\vspace*{\fill}
\begin{flushbottom}
 \begin{center}
 \tiny
 \textcopyright Andrew McIntosh
\end{center}
\end{flushbottom}



\newpage
\addtocounter{page}{5}
\vspace*{\fill}
\begingroup
\centering
\section*{The two F's}
\endgroup
\noindent Bids that have a minimum and maximum point range are called \textbf{limited}. Limited bids are \textbf{non forcing} (NF).\\

\noindent Bids that have a minimum number of HCP but not a maximum  are called \textbf{unlimited}. Unlimited bids are \textbf{forcing} (F).

\vspace*{\fill}
\begin{flushbottom}
 \begin{center}
 \tiny
 \textcopyright Andrew McIntosh
\end{center}
\end{flushbottom}


\newpage
\vspace*{\fill}
\begingroup
\centering
\section*{Opener's rebid: priority list}
\endgroup
\begin{itemize}
\item[(A)] Fit - raise
\item[(B)] Balanced - NT
\item[(C)] Unbalanced - no fit - show second suit
\item[(D)] Unbalanced - no fit - no second suit - repeat own suit
\end{itemize}


\vspace*{\fill}
\begin{flushbottom}
 \begin{center}
 \tiny
 \textcopyright Andrew McIntosh
\end{center}
\end{flushbottom}


\end{document}
